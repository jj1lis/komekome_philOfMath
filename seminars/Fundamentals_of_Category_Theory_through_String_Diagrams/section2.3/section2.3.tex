\documentclass[uplatex,a4paper,dvipdfmx,aspectratio=169,10pt]{beamer}

\usepackage{pxjahyper}
\usepackage{amsmath,amsfonts,amssymb,amsthm,bm,ascmac}
\usepackage{ascmac}
\usepackage{physics}
\usepackage{tikz}
\usepackage{gnuplot-lua-tikz}
\usetikzlibrary{intersections,calc,arrows.meta,cd,automata,positioning}
\usepackage{circuitikz}
\usepackage{here}
\usepackage{siunitx}
\usepackage{multicol,multirow}
\usepackage{systeme}
\usepackage[version=3]{mhchem}
\usepackage{chemfig}
\usepackage{url}
\usepackage{braket}
\usepackage{enumerate}
\usepackage{mathrsfs}
\usepackage{otf}
\usepackage{ulem}
\usepackage{stmaryrd}
\usepackage{listings}
\usepackage{bussproofs}
\usepackage{mathtools}
\usepackage{cite}
\usepackage{docmute}
\DeclareMathOperator{\Sinarc}{\mathrm{Sin}^{-1}}
\DeclareMathOperator{\Cosarc}{\mathrm{Cos}^{-1}}
\DeclareMathOperator{\Tanarc}{\mathrm{Tan}^{-1}}
\DeclareMathOperator{\Real}{\mathbb{R}}
\DeclareMathOperator{\Complex}{\mathbb{C}}
\DeclareMathOperator{\Rational}{\mathbb{Q}}
\DeclareMathOperator{\Natural}{\mathbb{N}}
\DeclareMathOperator{\Integer}{\mathbb{Z}}
\DeclareMathOperator{\Ker}{\mathrm{Ker}}
\DeclareMathOperator{\diffset}{\backslash}
\DeclareMathOperator{\define}{\overset{\text{def}}{\Longleftrightarrow}}
\newcommand{\typed}{\!:\!}
\newcommand{\Var} {\mathord{\mathbf{Var}}}
\newcommand{\TVar}{\mathord{\mathbf{TVar}}}
\newcommand{\Type}{\mathord{\mathbf{Type}}}
\newcommand{\Term}{\mathord{\mathbf{\Lambda}}}
\newcommand{\Asn}{\mathord{\mathbf{Asn}}}
\newcommand{\Cont}{\mathord{\mathbf{Cont}}}
\newcommand{\Church}{\mathord{\mathbf{M}}}
\newcommand{\Sub}  {\mathord{\mathrm{Sub}}}
\newcommand{\FV}  {\mathord{\mathrm{FV}}}
\newcommand{\BV}  {\mathord{\mathrm{BV}}}
\newcommand{\vars}{\mathcal{V}}
\newcommand{\tvars}{\mathbb{V}}
\newcommand{\terms}{\mathcal{T}}
\newcommand{\types}{\mathbb{T}}
\newcommand{\contexts}{\mathfrak{G}}
\newcommand{\ID}  {(\mathrm{ID})}
\newcommand{\WEAK}{(\mathrm{WEAK})}
\newcommand{\EXC}{(\mathrm{EXC})}
\newcommand{\VAR} {(\mathrm{VAR})}
\newcommand{\APP} {(\mathrm{APP})}
\newcommand{\BND} {(\mathrm{BND})}
\newcommand{\INST}{(\mathrm{INST})}
\newcommand{\GEN} {(\mathrm{GEN})}
\newcommand{\tbeta}{\to_\beta}
\newcommand{\ttbeta}{\twoheadrightarrow_\beta}
\newcommand{\dom}{\mathord{\mathrm{dom}}}
\newcommand{\cod}{\mathord{\mathrm{cod}}}
\newcommand{\evf}{{\mathrm{ev}}}
\newcommand{\Typ}{\mathord{\mathrm{Typ}}}
\newcommand{\SN}{\mathsf{SN}}
\newcommand{\SAT}{\mathsf{SAT}}
\newcommand{\Bool}{\mathbb{B}}
\newcommand{\TCP}{\mathbf{TCP}}
\newcommand{\SUP}{\mathbf{SUP}}
\newcommand{\Ltt}{\mathtt{L}}
\newcommand{\Rtt}{\mathtt{R}}
\newcommand{\Stt}{\mathtt{S}}
% \newcommand{\init}{\mathrm{init}}
\newcommand{\init}{0}
\newcommand{\acc}{\mathrm{acc}}
\newcommand{\rej}{\mathrm{rej}}
\newcommand{\bak}{\mathrm{bak}}
\newcommand{\textspace}{\text{\textvisiblespace}\,}
\newcommand{\terminate}{\!\downarrow}
\newcommand{\nterminate}{\!\uparrow}
\newcommand{\problems}{\mathcal{P}_\Sigma}
\newcommand{\classified}{\in}
\newcommand{\Halt}{\mathtt{Halt}}
\newcommand{\Ccal}{\mathcal{C}}
\newcommand{\Dcal}{\mathcal{D}}
\newcommand{\Ecal}{\mathcal{E}}
\newcommand{\onebf}{{\mathord{\mathbf{1}}}}
\newcommand{\twobf}{{\mathord{\mathbf{2}}}}
\newcommand{\onebb}{{\mathord{\mathit{1}}}}
\newcommand{\twobb}{{\mathord{\mathit{2}}}}
\newcommand{\Cat}{\mathbf{Cat}}
\newcommand{\CAT}{\mathbf{CAT}}

\DeclarePairedDelimiter{\interpret}{\llbracket}{\rrbracket}
\DeclarePairedDelimiter{\encode}{\langle}{\rangle}

\DeclareMathOperator{\betared}{\rightarrow_\beta}
\DeclareMathOperator{\lbetared}{\longrightarrow_\beta}
\DeclareMathOperator{\tbetared}{\twoheadrightarrow_\beta}
\newcommand{\nat}{\mathtt{nat}}
\newcommand{\snat}{\nat_\sigma}
\newcommand{\CSet}{{\mathbf{Set}}}
\newcommand{\DSet}{{\mathbf{DSet}}}
\DeclareMathOperator{\Hom}{{\mathrm{Hom}}}
\DeclareMathOperator{\ob}{{\mathrm{ob}}}
\DeclareMathOperator{\mor}{{\mathrm{mor}}}
\newcommand{\lamto}{{\lambda\!\!\to}}
\newcommand{\lamP}{\lambda\mathrm{P}}
\newcommand{\lamT}{\lambda 2}
\newcommand{\lamPT}{\lambda\mathrm{P}2}
\newcommand{\lamo}{\lambda\underline{\omega}}
\newcommand{\lamPo}{\lambda\mathrm{P}\underline{\omega}}
\newcommand{\lamO}{\lambda\omega}
\newcommand{\lamC}{\lambda\mathrm{C}}

\DeclareMathOperator{\pterm}{\mathbf{Pt}}
\newcommand{\PROP}{\textbf{PROP}}
\newcommand{\PRED}{\textbf{PRED}}
\newcommand{\PROPT}{\textbf{PROP2}}
\newcommand{\PREDT}{\textbf{PRED2}}
\newcommand{\PROPo}{\textbf{PROP}\underline{\omega}}
\newcommand{\PREDo}{\textbf{PRED}\underline{\omega}}
\newcommand{\PROPO}{\textbf{PROP}\omega}
\newcommand{\PREDO}{\textbf{PRED}\omega}

\newcommand{\Yes}{\mathtt{Yes}}
\newcommand{\No}{\mathtt{No}}

\theoremstyle{definition}
\setbeamertemplate{theorems}[numbered]
\newtheorem{dfn}{定義}
\newtheorem*{ndfn}{定義}
\newtheorem{axiom}{公理}
\newtheorem{thm}{定理}
\newtheorem{prop}{命題}
\newtheorem{lem}{補題}
\newtheorem{cor}{系}
\newtheorem{ex}{例}
\newtheorem{question}{問}
\newtheorem{caution}{注意}
\newtheorem{notation}{記法}
\newtheorem{conjecture}{予想}


\usepackage{pifont}
\lstset{
    language={C++}, %プログラミング言語によって変える。
    basicstyle={\ttfamily\small},
    %% breaklines=true, %折り返し
}
\lstset{
    basicstyle={\ttfamily},
    identifierstyle={\small},
    commentstyle={\smallitshape},
    keywordstyle={\color{blue}\small\bfseries},
    commentstyle={\color{gray}\small},
    stringstyle={\color{red}\small},
    tabsize=2,
    frame={tb},
    breaklines=true,
    columns=[l]{fullflexible},
    numbers=left,
    xrightmargin=0zw,
    xleftmargin=3zw,
    numberstyle={\scriptsize},
    stepnumber=1,
    numbersep=1zw,
    lineskip=-0.5ex
}

\newenvironment<>{varblock}[2][\textwidth]{%
    \setlength{\textwidth}{#1}
    \begin{actionenv}#3%
        \def\insertblocktitle{#2}%
        \par%
    \usebeamertemplate{block begin}}
    {\par%
        \usebeamertemplate{block end}%
\end{actionenv}}

\newenvironment<>{varalertblock}[2][\textwidth]{%
    \setlength{\textwidth}{#1}
    \begin{actionenv}#3%
        \def\insertblocktitle{#2}%
        \par%
    \usebeamertemplate{block alerted begin}}
    {\par%
        \usebeamertemplate{block alerted end}%
\end{actionenv}}

\newenvironment{scprooftree}[1]%
  {\gdef\scalefactor{#1}\begin{center}\proofSkipAmount \leavevmode}%
  {\scalebox{\scalefactor}{\DisplayProof}\proofSkipAmount \end{center} }

\usetheme{Madrid}
% \usecolortheme{crane}
\usecolortheme{default}
\useinnertheme{}
\useoutertheme{}
\renewcommand{\kanjifamilydefault}{\gtdefault}
\usefonttheme{professionalfonts}
\setbeamertemplate{items}[default]
\setbeamertemplate{navigation symbols}{}
\renewcommand{\baselinestretch}{1}
\setbeamercolor{alerted text}{fg=orange!50!red}

\newcommand{\vcenterinput}[1]{\vcenter{\hbox{\input{#1}}}}


\title{『ストリング図で学ぶ圏論の基礎』勉強会}
\subtitle{\S 2.3 関手圏}
\author{山田鈴太}
\institute[電通大院 M1]{電気通信大学大学院情報理工学研究科 博士前期課程1年}
\date{}

\renewcommand{\thefootnote}{*\arabic{footnote}}
\renewcommand{\theenumi}{(\arabic{enumi})}

\begin{document}
\begin{frame}
    \titlepage
\end{frame}

\begin{frame}[fragile]{2.3.1 関手圏の定義}
    \begin{itemize}
        \item 圏:対象とその間の射の集まり
    \end{itemize}
    \begin{center}
        \documentclass[uplatex,a4paper,dvipdfmx]{beamer}
\usepackage{pxjahyper}
\usepackage{amsmath,amsfonts,amssymb,amsthm,bm,ascmac}
\usepackage{ascmac}
\usepackage{physics}
\usepackage{tikz}
\usepackage{gnuplot-lua-tikz}
\usetikzlibrary{intersections,calc,arrows.meta,cd,automata,positioning}
\usepackage{circuitikz}
\usepackage{here}
\usepackage{siunitx}
\usepackage{multicol,multirow}
\usepackage{systeme}
\usepackage[version=3]{mhchem}
\usepackage{chemfig}
\usepackage{url}
\usepackage{braket}
\usepackage{enumerate}
\usepackage{mathrsfs}
\usepackage{otf}
\usepackage{ulem}
\usepackage{stmaryrd}
\usepackage{listings}
\usepackage{bussproofs}
\usepackage{mathtools}
\usepackage{cite}
\usepackage{docmute}
\DeclareMathOperator{\Sinarc}{\mathrm{Sin}^{-1}}
\DeclareMathOperator{\Cosarc}{\mathrm{Cos}^{-1}}
\DeclareMathOperator{\Tanarc}{\mathrm{Tan}^{-1}}
\DeclareMathOperator{\Real}{\mathbb{R}}
\DeclareMathOperator{\Complex}{\mathbb{C}}
\DeclareMathOperator{\Rational}{\mathbb{Q}}
\DeclareMathOperator{\Natural}{\mathbb{N}}
\DeclareMathOperator{\Integer}{\mathbb{Z}}
\DeclareMathOperator{\Ker}{\mathrm{Ker}}
\DeclareMathOperator{\diffset}{\backslash}
\DeclareMathOperator{\define}{\overset{\text{def}}{\Longleftrightarrow}}
\newcommand{\typed}{\!:\!}
\newcommand{\Var} {\mathord{\mathbf{Var}}}
\newcommand{\TVar}{\mathord{\mathbf{TVar}}}
\newcommand{\Type}{\mathord{\mathbf{Type}}}
\newcommand{\Term}{\mathord{\mathbf{\Lambda}}}
\newcommand{\Asn}{\mathord{\mathbf{Asn}}}
\newcommand{\Cont}{\mathord{\mathbf{Cont}}}
\newcommand{\Church}{\mathord{\mathbf{M}}}
\newcommand{\Sub}  {\mathord{\mathrm{Sub}}}
\newcommand{\FV}  {\mathord{\mathrm{FV}}}
\newcommand{\BV}  {\mathord{\mathrm{BV}}}
\newcommand{\vars}{\mathcal{V}}
\newcommand{\tvars}{\mathbb{V}}
\newcommand{\terms}{\mathcal{T}}
\newcommand{\types}{\mathbb{T}}
\newcommand{\contexts}{\mathfrak{G}}
\newcommand{\ID}  {(\mathrm{ID})}
\newcommand{\WEAK}{(\mathrm{WEAK})}
\newcommand{\EXC}{(\mathrm{EXC})}
\newcommand{\VAR} {(\mathrm{VAR})}
\newcommand{\APP} {(\mathrm{APP})}
\newcommand{\BND} {(\mathrm{BND})}
\newcommand{\INST}{(\mathrm{INST})}
\newcommand{\GEN} {(\mathrm{GEN})}
\newcommand{\tbeta}{\to_\beta}
\newcommand{\ttbeta}{\twoheadrightarrow_\beta}
\newcommand{\dom}{\mathord{\mathrm{dom}}}
\newcommand{\cod}{\mathord{\mathrm{cod}}}
\newcommand{\evf}{{\mathrm{ev}}}
\newcommand{\Typ}{\mathord{\mathrm{Typ}}}
\newcommand{\SN}{\mathsf{SN}}
\newcommand{\SAT}{\mathsf{SAT}}
\newcommand{\Bool}{\mathbb{B}}
\newcommand{\TCP}{\mathbf{TCP}}
\newcommand{\SUP}{\mathbf{SUP}}
\newcommand{\Ltt}{\mathtt{L}}
\newcommand{\Rtt}{\mathtt{R}}
\newcommand{\Stt}{\mathtt{S}}
% \newcommand{\init}{\mathrm{init}}
\newcommand{\init}{0}
\newcommand{\acc}{\mathrm{acc}}
\newcommand{\rej}{\mathrm{rej}}
\newcommand{\bak}{\mathrm{bak}}
\newcommand{\textspace}{\text{\textvisiblespace}\,}
\newcommand{\terminate}{\!\downarrow}
\newcommand{\nterminate}{\!\uparrow}
\newcommand{\problems}{\mathcal{P}_\Sigma}
\newcommand{\classified}{\in}
\newcommand{\Halt}{\mathtt{Halt}}
\newcommand{\Ccal}{\mathcal{C}}
\newcommand{\Dcal}{\mathcal{D}}
\newcommand{\Ecal}{\mathcal{E}}
\newcommand{\onebf}{{\mathord{\mathbf{1}}}}
\newcommand{\twobf}{{\mathord{\mathbf{2}}}}
\newcommand{\onebb}{{\mathord{\mathit{1}}}}
\newcommand{\twobb}{{\mathord{\mathit{2}}}}
\newcommand{\Cat}{\mathbf{Cat}}
\newcommand{\CAT}{\mathbf{CAT}}

\DeclarePairedDelimiter{\interpret}{\llbracket}{\rrbracket}
\DeclarePairedDelimiter{\encode}{\langle}{\rangle}

\DeclareMathOperator{\betared}{\rightarrow_\beta}
\DeclareMathOperator{\lbetared}{\longrightarrow_\beta}
\DeclareMathOperator{\tbetared}{\twoheadrightarrow_\beta}
\newcommand{\nat}{\mathtt{nat}}
\newcommand{\snat}{\nat_\sigma}
\newcommand{\CSet}{{\mathbf{Set}}}
\newcommand{\DSet}{{\mathbf{DSet}}}
\DeclareMathOperator{\Hom}{{\mathrm{Hom}}}
\DeclareMathOperator{\ob}{{\mathrm{ob}}}
\DeclareMathOperator{\mor}{{\mathrm{mor}}}
\newcommand{\lamto}{{\lambda\!\!\to}}
\newcommand{\lamP}{\lambda\mathrm{P}}
\newcommand{\lamT}{\lambda 2}
\newcommand{\lamPT}{\lambda\mathrm{P}2}
\newcommand{\lamo}{\lambda\underline{\omega}}
\newcommand{\lamPo}{\lambda\mathrm{P}\underline{\omega}}
\newcommand{\lamO}{\lambda\omega}
\newcommand{\lamC}{\lambda\mathrm{C}}

\DeclareMathOperator{\pterm}{\mathbf{Pt}}
\newcommand{\PROP}{\textbf{PROP}}
\newcommand{\PRED}{\textbf{PRED}}
\newcommand{\PROPT}{\textbf{PROP2}}
\newcommand{\PREDT}{\textbf{PRED2}}
\newcommand{\PROPo}{\textbf{PROP}\underline{\omega}}
\newcommand{\PREDo}{\textbf{PRED}\underline{\omega}}
\newcommand{\PROPO}{\textbf{PROP}\omega}
\newcommand{\PREDO}{\textbf{PRED}\omega}

\newcommand{\Yes}{\mathtt{Yes}}
\newcommand{\No}{\mathtt{No}}

\theoremstyle{definition}
\setbeamertemplate{theorems}[numbered]
\newtheorem{dfn}{定義}
\newtheorem*{ndfn}{定義}
\newtheorem{axiom}{公理}
\newtheorem{thm}{定理}
\newtheorem{prop}{命題}
\newtheorem{lem}{補題}
\newtheorem{cor}{系}
\newtheorem{ex}{例}
\newtheorem{question}{問}
\newtheorem{caution}{注意}
\newtheorem{notation}{記法}
\newtheorem{conjecture}{予想}


\usepackage{pifont}
\lstset{
    language={C++}, %プログラミング言語によって変える。
    basicstyle={\ttfamily\small},
    %% breaklines=true, %折り返し
}
\lstset{
    basicstyle={\ttfamily},
    identifierstyle={\small},
    commentstyle={\smallitshape},
    keywordstyle={\color{blue}\small\bfseries},
    commentstyle={\color{gray}\small},
    stringstyle={\color{red}\small},
    tabsize=2,
    frame={tb},
    breaklines=true,
    columns=[l]{fullflexible},
    numbers=left,
    xrightmargin=0zw,
    xleftmargin=3zw,
    numberstyle={\scriptsize},
    stepnumber=1,
    numbersep=1zw,
    lineskip=-0.5ex
}

\newenvironment<>{varblock}[2][\textwidth]{%
    \setlength{\textwidth}{#1}
    \begin{actionenv}#3%
        \def\insertblocktitle{#2}%
        \par%
    \usebeamertemplate{block begin}}
    {\par%
        \usebeamertemplate{block end}%
\end{actionenv}}

\newenvironment<>{varalertblock}[2][\textwidth]{%
    \setlength{\textwidth}{#1}
    \begin{actionenv}#3%
        \def\insertblocktitle{#2}%
        \par%
    \usebeamertemplate{block alerted begin}}
    {\par%
        \usebeamertemplate{block alerted end}%
\end{actionenv}}

\newenvironment{scprooftree}[1]%
  {\gdef\scalefactor{#1}\begin{center}\proofSkipAmount \leavevmode}%
  {\scalebox{\scalefactor}{\DisplayProof}\proofSkipAmount \end{center} }

\usetheme{Madrid}
% \usecolortheme{crane}
\usecolortheme{default}
\useinnertheme{}
\useoutertheme{}
\renewcommand{\kanjifamilydefault}{\gtdefault}
\usefonttheme{professionalfonts}
\setbeamertemplate{items}[default]
\setbeamertemplate{navigation symbols}{}
\renewcommand{\baselinestretch}{1}
\setbeamercolor{alerted text}{fg=orange!50!red}

\newcommand{\vcenterinput}[1]{\vcenter{\hbox{\input{#1}}}}

\begin{document}
\begin{tikzpicture}[>=stealth, auto, node distance=2cm]
    \draw[rounded corners=2pt] (0, 0) rectangle (5, 3);
    \node[fill=white] at (1, 3) {$\Ccal$};

    \node (p0) at (0.5, 2) {$\bullet$};
    \node (p1) at (1, 0.5) {$\bullet$};
    \node (p2) at (2, 1.8) {$\bullet$};
    \node (p3) at (3, 1) {$\bullet$};
    \node (p4) at (4, 2.5) {$\bullet$};
    \node (p5) at (4.5, 0.7) {$\bullet$};

    \draw[thick, ->] (p0) to [bend left] (p1);
    \draw[thick, ->] (p0) to [bend right] (p1);
    \draw[thick, ->] (p3) to [bend left] (p2);
    \draw[thick, ->] (p2) to [bend left] (p3);
    \draw[thick, ->] (p2) -- (p0);
    \draw[thick, ->] (p2) -- (p4);
    \draw[thick, ->] (p0) to [bend left] (p4);

    \draw[dashed] (p4) -- (6, 2) node[right]{対象};
    \draw[dashed] (3, 2) -- (6, 1) node[right]{射};
\end{tikzpicture}
\end{document}

    \end{center}
    \begin{itemize}
        \item 自然変換:関手から関手への矢印
    \end{itemize}
    \begin{equation*}
        \begin{tikzcd}[sep=huge]
            \Ccal \arrow[r, bend left=35, "F"] \arrow[r, bend left=30, phantom, ""' name=F1] \arrow[r, bend right=35, "G"'] \arrow[r, bend right=30, phantom, "" name=F2] \arrow[Rightarrow, from=F1, to=F2, "\alpha"] & \Dcal
        \end{tikzcd}
    \end{equation*}
    \begin{itemize}
        \item 関手を対象,自然変換を射とする圏が作れるのでは?
    \end{itemize}
\end{frame}

\begin{frame}[fragile]{2.3.1 関手圏の定義}
    % $F, G \colon \Ccal \to \Dcal$は関手
    \begin{block}{関手圏}
        $\Ccal, \Dcal$を圏とする.

        次の通り定義される圏を$\Ccal$から$\Dcal$への\alert{関手圏}といい,$\Dcal^\Ccal$で表す.
        \begin{itemize}
            \item 対象は各関手$\Ccal \to \Dcal$
            \item 対象$F, G \colon \Ccal \to \Dcal$に対して,$F$から$G$への射は各自然変換$\alpha \colon F \Rightarrow G$
            \item 各射 \begin{tikzcd}[cramped] F \arrow[r, "\alpha"] & G \arrow[r, "\beta"] & H \end{tikzcd} に対して,その合成$\beta \alpha$は自然変換の垂直合成
            \item 各対象$F$について,その恒等射は恒等自然変換$1_F$
        \end{itemize}
    \end{block}
    \begin{equation*}
        \underbrace{\vcenterinput{figures/figure02.tex}}_{\text{圏$\Dcal^\Ccal$の射$\alpha\colon F \to G$}}\ 
            =\ \underbrace{\vcenterinput{figures/figure01.tex}}_{\text{関手$\Ccal \to \Dcal$の間の自然変換$\alpha \colon F \Rightarrow G$}}
    \end{equation*}
\end{frame}

\begin{frame}[fragile]{2.3.1 関手圏の定義}
    \begin{exampleblock}{演習問題2.3.1}
        関手圏が圏であることを示せ.
    \end{exampleblock}
    \fbox{証明}\\
    $\Ccal, \Dcal$を任意の圏として,$\Dcal^\Ccal$が圏の性質を満たすことを確認する.
    \begin{description}[射の合成の結合律:]
        \item[射の合成の閉性:] 任意の自然変換\begin{tikzcd} F \arrow[r, Rightarrow, "\alpha"] & G \arrow[r, Rightarrow, "\beta"] & H \end{tikzcd}に対して,\\
            垂直合成$\beta \alpha$は再び自然変換$F \Rightarrow H$である (cf. \S 2.1.1).
        \item[射の合成の結合律:] 任意の自然変換\begin{tikzcd} F \arrow[r, Rightarrow, "\alpha"] & G \arrow[r, Rightarrow, "\beta"] & H \arrow[r, Rightarrow, "\gamma"] & I \end{tikzcd}に対して,\\
            結合律が次の通り成り立つ:
    \end{description}
    \begin{align*}
        \gamma(\beta\alpha) &= \qty{\gamma_a (\beta_a \alpha_a)}_{a \in \Ccal} \\
                            &= \qty{(\gamma_a\beta_a) \alpha_a}_{a \in \Ccal} \\
                            &= (\gamma \beta) \alpha.
    \end{align*}
\end{frame}
\begin{frame}[fragile]{2.3.1 関手圏の定義}
    \begin{exampleblock}{演習問題2.3.1}
        関手圏が圏であることを示せ.
    \end{exampleblock}
    \begin{description}
        \item[恒等射の性質:] 任意の自然変換\begin{tikzcd}F \arrow[r, Rightarrow, "\alpha"] & G\end{tikzcd}に対して,$\alpha 1_F = \alpha = 1_G \alpha$が成り立つ (cf. \S 2.1.1).
    \end{description}
    したがって$\Dcal^\Ccal$は圏である.\qed
\end{frame}


\begin{frame}[fragile]{2.3.1 関手圏の定義}
    % $\Ccal, \Dcal$は圏,$F, G \colon \Ccal \to \Dcal$は関手
    % \begin{itemize}
    射$\alpha \in \Dcal^\Ccal(F, G)$が同型射\ $\Longleftrightarrow$\ 自然変換$\alpha$が自然同型
    % \end{itemize}
    \begin{description}
        \item[($\Rightarrow$)] 
            \begin{itemize}
                \item $\alpha$が同型射なら,逆射$\alpha^{-1} \colon G \to F$が存在して$\alpha^{-1} \circ \alpha = 1_F$
                \item 各$a \in \Ccal$について$\alpha^{-1}_a \circ \alpha_a = 1_{Fa}$
                    \begin{itemize}
                        \item $\alpha_a \circ \alpha^{-1}_a = 1_{Ga}$も同様
                    \end{itemize}
                \item したがって各$a \in \Ccal$について$\alpha_a$が同型射 $\define$ $\alpha$は自然同型
            \end{itemize}
        \item[($\Leftarrow$)]
            \begin{itemize}
                \item $\alpha$が自然同型なら,各$a \in \Ccal$について$\alpha_a$が同型射
                \item 逆射$\alpha^{-1}_a$を集めて自然変換$\alpha^{-1} = \qty{\alpha^{-1}_a}_{a \in \Ccal}$を構成
                \item $\alpha^{-1} \circ \alpha = 1_F$と$\alpha \circ \alpha^{-1} = 1_G$が成り立つ($\Rightarrow$と同じ議論)
                \item よって$\alpha$は同型射
            \end{itemize}
    \end{description}
\end{frame}

\begin{frame}[fragile]{2.3.1 関手圏の定義}
    $\Ccal, \Dcal$が局所小圏であっても,$\Dcal^\Ccal$も局所小圏とは限らない.
    \begin{exampleblock}{演習問題2.3.2}
        \begin{enumerate}[(a)]
            \item 圏$\Cat$は局所小圏であることを示せ.
            \item 小圏$\Ccal$と局所小圏$\Dcal$について関手圏$\Dcal^\Ccal$は局所小圏であることを示せ.
            \item $\Ccal, \Dcal$が共に局所小圏であるとき,関手圏$\Dcal^\Ccal$が必ずしも局所小圏でないことを示せ.
        \end{enumerate}
    \end{exampleblock}
    \begin{enumerate}[(a)]
        \item いま$\Ccal, \Dcal \in \Cat$を固定して,$\Cat(\Ccal, \Dcal)$が集合であるか考える.
            各射$F \in \Cat(\Ccal, \Dcal)$は関手$\Ccal \to \Dcal$であり,関手は
            \begin{itemize}
                \item 対象への作用$\ob \Ccal \to \ob \Dcal$
                \item 射への作用$\mor \Ccal \to \mor \Dcal$
            \end{itemize}
            から成る.$\Ccal, \Dcal$は小圏だから,$\ob \Ccal, \mor \Ccal$ (resp. $\Dcal$) は集合.
            したがって$\ob \Dcal^{\ob \Ccal} \times \mor \Dcal^{\mor \Ccal}$とでも書くべき作用の組全体も集合をなすので,
            その部分クラスである$\Cat(\Ccal, \Dcal)$も集合である.
    \end{enumerate}
\end{frame}

\begin{frame}[fragile]{2.3.1 関手圏の定義}
    \begin{exampleblock}{演習問題2.3.2}
        \begin{enumerate}[(a)]
            \item 圏$\Cat$は局所小圏であることを示せ.
            \item 小圏$\Ccal$と局所小圏$\Dcal$について関手圏$\Dcal^\Ccal$は局所小圏であることを示せ.
            \item $\Ccal, \Dcal$が共に局所小圏であるとき,関手圏$\Dcal^\Ccal$が必ずしも局所小圏でないことを示せ.
        \end{enumerate}
    \end{exampleblock}
    \begin{enumerate}[(a)]
        \setcounter{enumi}{1}
        \item 各対象$F, G$に対して,$\Dcal^\Ccal(F, G)$が集合であるか考える.
            各射$\alpha \in \Dcal^\Ccal(F, G)$は自然変換$F \Rightarrow G$であり,自然変換は対象への作用$\ob \Ccal \to \mor \Dcal$のみから決定されることを思い出す(cf. \S 1.3.2).
            条件より$\ob \Ccal, \mor \Dcal$は共に集合であるから,対象への作用全体も集合をなし,その部分クラスである$\Dcal^\Ccal(F, G)$も集合となる.
        \item $\Ccal$が単に局所小である場合,$\ob \Ccal$は集合であるとは限らないので,上の議論が成り立たない.

            (具体的な反例は次ページ)
    \end{enumerate}
\end{frame}

\begin{frame}[fragile]{2.3.1 関手圏の定義}
    \begin{exampleblock}{演習問題2.3.2}
        \begin{enumerate}[(c)]
            \item $\Ccal, \Dcal$が共に局所小圏であるとき,関手圏$\Dcal^\Ccal$が必ずしも局所小圏でないことを示せ.
        \end{enumerate}
    \end{exampleblock}
    \begin{itemize}
        \item 記号を次のように定める
            \begin{itemize}
                \item $\DSet$ : 全ての集合を対象とする離散圏
                \item $\onebb$ : 1元集合$\qty{*}$
                \item $\twobb$ : 2元集合$\qty{0, 1}$
                \item $F\colon \DSet \to \CSet$ : $\DSet$の各対象を$\onebb$に,各射を$1_\onebb$に写す関手
                \item $G\colon \DSet \to \CSet$ : $\DSet$の各対象を$\twobb$に,各射を$1_\twobb$に写す関手
            \end{itemize}
        \item 自然変換$\alpha\colon F \Rightarrow G$は$\CSet$の射(つまり写像)の族$\qty{\alpha_X \colon \onebb \to \twobb}_{X \in \DSet}$
        \item 各$\alpha_X$は行き先を$0$にするか$1$にするかの選択
            \begin{itemize}
                \item $\alpha$ごとに集合族$P_\alpha := \Set{ X \in \DSet | \alpha_X(*) = 1}$がある
            \end{itemize}
        \item 自然変換$F \Rightarrow G$全体$\CSet^\DSet(F, G)$のサイズは「集合全体のクラスの羃クラス」に等しい
        \item これは集合でないので,$\CSet^\DSet$は局所小にならない$\leftarrow$反例!
    \end{itemize}
\end{frame}

\begin{frame}[fragile]{2.3.1 関手圏の定義}
    $\Ccal$: 圏
    \begin{exampleblock}{\fbox{例2.23} 関手圏$\Ccal^\onebf$}
        関手圏$\Ccal^\onebf$の対象および射は,それぞれ圏$\Ccal$の対象および射と同一視できる.
    \end{exampleblock}
    \fbox{例1.49} 任意の対象は関手とみなせる:各$a \in \Ccal$に対して$\Delta_{\onebf}a \colon \onebf \to \Ccal;\, * \mapsto a$

    \fbox{例1.72} 任意の射は自然変換とみなせる:各$f \in \Ccal(a, b)$に対して$\qty{f}_{* \in \onebb} \colon a \Rightarrow b$

    \begin{itemize}
        \item $\Ccal^\onebf$の対象は関手$\Delta_{\onebf}a$
            \begin{itemize}
                \item $a \in \Ccal$と同一視
            \end{itemize}
        \item $\Ccal^\onebf$の射は自然変換$\qty{f}_{* \in \onebf}\colon a \Rightarrow b$
            \begin{itemize}
                \item $f \colon a \to b$と同一視
            \end{itemize}
    \end{itemize}
    したがって関手$F \colon \Ccal^\onebf \to \Ccal; \Delta_{\onebf}a \mapsto a$ によって同型$\Ccal^\onebf \cong \Ccal$を得る.
\end{frame}

\begin{frame}[fragile]{2.3.1 関手圏の定義}
    $\Ccal$: 圏
    \begin{exampleblock}{\fbox{例2.24} 関手圏$\Ccal^\twobf$}
        関手圏$\Ccal^\twobf$の対象は,圏$\Ccal$の射と同一視できる.$\Ccal^\twobf$を\structure{射圏} (arrow category) とも呼ぶ.
    \end{exampleblock}
    \begin{equation*}
        \twobf := \quad \begin{tikzcd}
            1 \arrow[loop left, "1_1"] \arrow[r, "!"] & 2 \arrow[loop right, "1_2"]
        \end{tikzcd}
    \end{equation*}
    \begin{itemize}
        \item $\Ccal^\twobf$の対象$F$は関手$\twobf \to \Ccal$
        \item $F$は射$! \in \mor \twobf$の行き先$F! \in \mor \Ccal$のみによって決まる
            \begin{itemize}
                \item $F! := f$と決めたなら,$F1 := \dom f,\, F2 := \cod f$とせざるを得ない
            \end{itemize}
        \item 逆に任意の射$f \in \mor \Ccal$に対して,$F! = f$となる関手$F$を1つだけ作れる
        \item よって$\Ccal^\twobf$の対象は$\Ccal$の射と一対一に対応している
    \end{itemize}
\end{frame}

\begin{frame}[fragile]{2.3.1 関手圏の定義}
    \begin{exampleblock}{\fbox{例2.24} 関手圏$\Ccal^\twobf$}
        関手圏$\Ccal^\twobf$の対象は,圏$\Ccal$の射と同一視できる.$\Ccal^\twobf$を射圏 (arrow category) とも呼ぶ.
    \end{exampleblock}
    \begin{itemize}
        \item $F, G \colon \twobf \to \Ccal$に対して$f := F!, g := G!$とする
        \item $\Ccal^\twobf$の射$\alpha \in \Ccal^\twobf(F, G)$は自然変換$F \Rightarrow G$
            \begin{itemize}
                \item $g \alpha_1 = \alpha_2 f$を満たす各射$\alpha_1 \colon F1 \to G1,\, \alpha_2 \colon F2 \to G2$から自然変換$\alpha = \qty{\alpha_1, \alpha_2}$を作れる
            \end{itemize}
    \end{itemize}
    \begin{gather*}
        \vcenterinput{figures/figure03.tex} \ = \ \vcenterinput{figures/figure04.tex}
    \end{gather*}
\end{frame}

\begin{frame}[fragile]{2.3.2 関手圏に関する関手の例}
    $\Ccal, \Dcal, \Ecal$: 圏

    % 関手圏から関手圏への関手として「関手$F$を前から水平合成する」「関手$F$を後ろから水平合成する」という操作を考える.
    \begin{exampleblock}{\fbox{例2.25} 関手を前から施す関手}
        関手$F \colon \Ccal \to \Dcal$に対して,関手$- \bullet F \colon \Ecal^\Dcal \to \Ecal^\Ccal$を次で定義する.
        \begin{description}[対象への作用:]
            \item[対象への作用:]
                \begin{equation*}
                    \begin{array}{ccc}
                        \ob \Ecal^\Dcal & \to & \ob \Ecal^\Ccal  \\
                        % G \colon \Dcal \to \Ecal & \mapsto & G \bullet F \colon \Ccal \to \Ecal.
                        \underbrace{G}_{\Dcal \to \Ecal} & \mapsto & \underbrace{G \bullet F}_{\Ccal \to \Ecal}.
                    \end{array}
                \end{equation*}
            \item[射への作用:] 各対象(すなわち関手)$G, H \in \Ecal^\Dcal$について,
                \begin{equation*}
                    \begin{array}{ccc}
                        \Ecal^\Dcal(G, H) & \to & \Ecal^\Ccal(G \bullet F, H \bullet F)  \\
                        % \alpha \colon G \Rightarrow H & \mapsto & \alpha \bullet F \colon G \bullet F \Rightarrow H \bullet F.
                        \underbrace{\alpha}_{G \Rightarrow H} & \mapsto & \underbrace{\alpha \bullet F}_{G \bullet F \Rightarrow H \bullet F}.
                    \end{array}
                \end{equation*}
        \end{description}
    \end{exampleblock}
    $\alpha \bullet F := \alpha \bullet 1_F = \qty{\alpha \bullet 1_{Fa}}_{a \in \Ccal}$に注意
\end{frame}

\begin{frame}[fragile]{2.3.2 関手圏に関する関手の例}
    $\Ccal, \Dcal, \Ecal$: 圏,$F$: 関手$\Ccal \to \Dcal$

    % 関手圏から関手圏への関手として「関手$F$を前から水平合成する」「関手$F$を後ろから水平合成する」という操作を考える.
    \begin{exampleblock}{\fbox{例2.25} 関手を前から施す関手}
        \begin{equation*}
            \begin{array}{ccc}
                \Ecal^\Dcal(G, H) & \to & \Ecal^\Ccal(G \bullet F, H \bullet F)  \\
                        % \alpha \colon G \Rightarrow H & \mapsto & \alpha \bullet F \colon G \bullet F \Rightarrow H \bullet F.
                \underbrace{\alpha}_{G \Rightarrow H} & \mapsto & \underbrace{\alpha \bullet F}_{G \bullet F \Rightarrow H \bullet F}.
            \end{array}
        \end{equation*}
    \end{exampleblock}
    \begin{equation*}
        \vcenterinput{figures/figure05.tex} \xmapsto{- \bullet F} \vcenterinput{figures/figure06.tex} = \vcenterinput{figures/figure07.tex}\tag{2.26}
    \end{equation*}
\end{frame}

\begin{frame}[fragile]{2.3.2 関手圏に関する関手の例}
    $\Ccal, \Dcal, \Ecal$: 圏,$F$: 関手$\Ccal \to \Dcal$

    % 関手圏から関手圏への関手として「関手$F$を前から水平合成する」「関手$F$を後ろから水平合成する」という操作を考える.
    \begin{exampleblock}{演習問題2.3.5(前半)}
        例2.25で定めた写像$- \bullet F \colon \Ecal^\Dcal \to \Ecal^\Ccal$が関手であることを示せ.
    \end{exampleblock}
    \fbox{証明}

    $\Ecal^\Dcal$の射(すなわち自然変換)\begin{tikzcd}[cramped] G \arrow[r, Rightarrow, "\alpha"] & H \arrow[r, Rightarrow, "\beta"] & I \end{tikzcd} を任意に取って,関手の公理
    \begin{equation*}
        (\beta \alpha) \bullet F = (\beta \bullet F) \circ (\alpha \bullet F)
    \end{equation*}
    を満たすことを次の通り確認できる:
    \begin{align*}
        (\beta \alpha) \bullet F &:= (\beta \circ \alpha) \bullet 1_F \\
                                 &= (\beta \circ \alpha) \bullet (1_F \circ 1_F) \\
                                 &= (\beta \bullet 1_F) \circ (\alpha \bullet 1_F) \tag{$\because$ 命題2.9}\\
                                 &=: (\beta \bullet F) \circ (\alpha \bullet F).
    \end{align*}
    したがって$- \bullet F$は関手である.\qed
\end{frame}

\begin{frame}[fragile]{2.3.2 関手圏に関する関手の例}
    $\Ccal, \Dcal, \Ecal$: 圏,$F, G$: 関手$\Ccal \to \Dcal$
    \begin{exampleblock}{\fbox{例2.25} 関手を前から施す関手}
        同様に自然変換$\gamma \colon F \Rightarrow G$に対して,自然変換$- \bullet \gamma \colon (- \bullet F) \Rightarrow (- \bullet G)$が定義できる.
        \begin{description}[対象への作用:]
            \item[対象への作用:]
                \begin{equation*}
                    \begin{array}{ccc}
                        \ob \Ecal^\Dcal & \to & \mor \Ecal^{\Ccal}  \\
                        % G \colon \Dcal \to \Ecal & \mapsto & G \bullet F \colon \Ccal \to \Ecal.
                        \underbrace{H}_{\Dcal \to \Ecal} & \mapsto & \underbrace{H \bullet \gamma}_{(H \bullet F) \to (H \bullet G)}.
                    \end{array}
                \end{equation*}
            \item[射への作用:] 各対象(すなわち関手)$H, I \in \Ecal^\Dcal$に対して,
                \begin{equation*}
                    \begin{array}{ccc}
                        \Ecal^\Dcal(H, I)& \to & \Ecal^\Ccal(H \bullet \gamma, I \bullet \gamma)  \\
                        % G \colon \Dcal \to \Ecal & \mapsto & G \bullet F \colon \Ccal \to \Ecal.
                        \underbrace{\alpha}_{H \Rightarrow I} & \mapsto & \underbrace{\alpha \bullet \gamma}_{(H \bullet \gamma) \Rightarrow (I \bullet \gamma)}.
                    \end{array}
                \end{equation*}
        \end{description}
    \end{exampleblock}
\end{frame}

\begin{frame}[fragile]{2.3.2 関手圏に関する関手の例}
    $\Ccal, \Dcal, \Ecal$: 圏,$F, G$: 関手$\Ccal \to \Dcal$,$\gamma$: 自然変換$F \Rightarrow G$
    \begin{exampleblock}{演習問題2.3.5(後半)}
        例2.25で定めた写像$- \bullet \gamma \colon (- \bullet F) \Rightarrow (- \bullet G)$が自然変換であることを示せ.
                \begin{equation*}
                    \begin{array}{ccc}
                        \ob \Ecal^\Dcal & \to & \mor \Ecal^{\Ccal}  \\
                        % G \colon \Dcal \to \Ecal & \mapsto & G \bullet F \colon \Ccal \to \Ecal.
                        \underbrace{H}_{\Dcal \to \Ecal} & \mapsto & \underbrace{H \bullet \gamma}_{(H \bullet F) \to (H \bullet G)}.
                    \end{array}
                \end{equation*}
    \end{exampleblock}
    \fbox{証明}

    $\Ecal^\Dcal$の射(すなわち自然変換)\begin{tikzcd}[cramped] H \arrow[r, Rightarrow, "\alpha"] & I \end{tikzcd} を任意に取って,
    自然性の条件
    \begin{equation*}
        (\alpha \bullet G) \circ (H \bullet \gamma) = (I \bullet \gamma) \circ (\alpha \bullet F)
    \end{equation*}
    を満たすことを示す.
\end{frame}

\begin{frame}[fragile]{2.3.2 関手圏に関する関手の例}
    $\Ccal, \Dcal, \Ecal$: 圏,$F, G, H, I$: 関手$\Ccal \to \Dcal$,$\gamma$: 自然変換$F \Rightarrow G$,$\alpha$: 自然変換$H \Rightarrow I$
    \begin{exampleblock}{演習問題2.3.5(後半)}
        例2.25で定めた写像$- \bullet \gamma \colon (- \bullet F) \Rightarrow (- \bullet G)$が自然変換であることを示せ.
    \end{exampleblock}
    \begin{align*}
        (\alpha \bullet G) \circ (H \bullet \gamma) &:= (\alpha \bullet 1_G) \circ (1_H \bullet \gamma) \\
                                                     &= (\alpha \circ 1_H) \bullet (1_G \circ \gamma) \tag{$\because$命題2.9} \\
                                                     &= (1_I \circ \alpha) \bullet (\gamma \circ 1_F) \\
                                                     &= (1_I \bullet \gamma) \circ (\alpha \bullet 1_F) \tag{$\because$命題2.9} \\
                                                     &=: (I \bullet \gamma) \circ (\alpha \bullet F).
    \end{align*}
    したがって,$- \bullet \gamma$は自然変換である.\qed
\end{frame}

\begin{frame}[fragile]{2.3.2 関手圏に関する関手の例}
    $\Ccal, \Dcal, \Ecal$: 圏,$F, G$: 関手$\Ccal \to \Dcal$
    \begin{exampleblock}{\fbox{例2.25} 関手を前から施す関手}
        同様に自然変換$\gamma \colon F \Rightarrow G$に対して,自然変換$- \bullet \gamma \colon (- \bullet F) \Rightarrow (- \bullet G)$が定義できる.
    \end{exampleblock}
    自然変換$\theta \colon G \Rightarrow H$と$\tau \colon F' \Rightarrow G'$に対して
    \begin{equation*}
        \begin{aligned}
            (- \bullet \theta) (- \bullet \gamma) &= - \bullet \theta\gamma, \\
            (- \bullet \tau) \bullet (- \bullet \gamma) &= - \bullet (\tau \bullet \gamma)
        \end{aligned}
        \tag{2.27}
    \end{equation*}
    が成り立つ.
\end{frame}

\begin{frame}[fragile]{2.3.2 関手圏に関する関手の例}
    $\Ccal, \Dcal, \Ecal$: 圏

    当然,例2.25の逆も考えられる.
    \begin{exampleblock}{\fbox{例2.28} 関手を後ろから施す関手}
        関手$F \colon \Dcal \to \Ecal$に対して,関手$F \bullet - \colon \Dcal^\Ccal \to \Ecal^\Ccal$を次で定義する.
        \begin{description}[対象への作用:]
            \item[対象への作用:]
                \begin{equation*}
                    \begin{array}{ccc}
                        \ob \Dcal^\Ccal & \to & \ob \Ecal^\Ccal  \\
                        % G \colon \Dcal \to \Ecal & \mapsto & G \bullet F \colon \Ccal \to \Ecal.
                        \underbrace{G}_{\Ccal \to \Dcal} & \mapsto & \underbrace{F \bullet G}_{\Ccal \to \Ecal}.
                    \end{array}
                \end{equation*}
            \item[射への作用:] 各対象(すなわち関手)$G, H \in \Dcal^\Ccal$について,
                \begin{equation*}
                    \begin{array}{ccc}
                        \Dcal^\Ccal(G, H) & \to & \Ecal^\Ccal(F \bullet G, F \bullet H)  \\
                        % \alpha \colon G \Rightarrow H & \mapsto & \alpha \bullet F \colon G \bullet F \Rightarrow H \bullet F.
                        \underbrace{\alpha}_{G \Rightarrow H} & \mapsto & \underbrace{F \bullet \alpha}_{F \bullet G \Rightarrow F \bullet H}.
                    \end{array}
                \end{equation*}
        \end{description}
    \end{exampleblock}
\end{frame}
\begin{frame}[fragile]{2.3.2 関手圏に関する関手の例}
    $\Ccal, \Dcal, \Ecal$: 圏,$F, G$: 関手$\Dcal^\Ccal \to \Ecal^\Ccal$
    \begin{exampleblock}{\fbox{例2.28} 関手を後ろから施す関手}
        自然変換$\gamma \colon F \Rightarrow G$に対して,自然変換$\gamma \bullet - \colon (F \bullet -) \Rightarrow (G \bullet -)$が定義できる.
        \begin{description}[対象への作用:]
            \item[対象への作用:]
                \begin{equation*}
                    \begin{array}{ccc}
                        \ob \Dcal^\Ccal & \to & \mor \Ecal^{\Ccal}  \\
                        % G \colon \Dcal \to \Ecal & \mapsto & G \bullet F \colon \Ccal \to \Ecal.
                        \underbrace{H}_{\Ccal \to \Dcal} & \mapsto & \underbrace{\gamma \bullet H}_{(F \bullet H) \to (G \bullet H)}.
                    \end{array}
                \end{equation*}
            % \item[射への作用:] 各対象(すなわち関手)$H, I \in \Ecal^\Dcal$に対して,
            %     \begin{equation*}
            %         \begin{array}{ccc}
            %             \Ecal^\Dcal(H, I)& \to & \Ecal^\Ccal(H \bullet \gamma, I \bullet \gamma)  \\
            %             % G \colon \Dcal \to \Ecal & \mapsto & G \bullet F \colon \Ccal \to \Ecal.
            %             \underbrace{\alpha}_{H \Rightarrow I} & \mapsto & \underbrace{\alpha \bullet \gamma}_{(H \bullet \gamma) \Rightarrow (I \bullet \gamma)}.
            %         \end{array}
            %     \end{equation*}
        \end{description}
    \end{exampleblock}
    自然変換$\theta \colon G \Rightarrow H$と$\tau \colon F' \Rightarrow G'$に対して
    \begin{equation*}
        \begin{aligned}
            (\theta \bullet -) (\gamma \bullet -) &= \theta\gamma \bullet -, \\
            (\tau \bullet -) \bullet (\gamma \bullet -) &= (\tau \bullet \gamma) \bullet -
        \end{aligned}
        \tag{2.29}
    \end{equation*}
    が成り立つ.
\end{frame}

\begin{frame}[fragile]{2.3.2 関手圏に関する関手の例}
    $\Ccal, \Dcal, \Ecal$: 圏

    % 関手圏から関手圏への関手として「関手$F$を前から水平合成する」「関手$F$を後ろから水平合成する」という操作を考える.
    \fbox{例2.25} 関手$F \colon \Ccal \to \Dcal$に対して,関手$- \bullet F \colon \Ecal^\Dcal \to \Ecal^\Ccal$を定義
    \begin{exampleblock}{\fbox{例2.31} 評価関手}
        例2.25で$\Ccal = \onebf$と取る.
        \begin{itemize}
            \item $F \colon \onebf \to \Dcal$は対象$d = F(*) \in \Dcal$と同一視できる (cf. 例1.49)
            \item $\Ecal^\onebf$は$\Ecal$と同型 (cf. 例2.23)
        \end{itemize}
        この関手$- \bullet d \colon \Ecal^\Dcal \to \Ecal$を\structure{評価関手} (evaluation functor) と呼び,$\evf_d$でも表す.
    \end{exampleblock}
    すなわち,この$\evf_d$は次のような関手.
        \begin{description}[対象への作用:]
            \item[対象への作用:] 各対象(すなわち関手)$G \in \Ecal^\Dcal$を$G \bullet d = Gd \in \Ecal$に写す
            \item[射への作用:] 各射(すなわち自然変換)\begin{tikzcd}[cramped] G \arrow[r, Rightarrow, "\alpha"] & H \end{tikzcd} を$\alpha \bullet d = \alpha_d \in \Ecal(Gd, Hd)$に写す.
        \end{description}
\end{frame}

\begin{frame}[fragile]{2.3.2 関手圏に関する関手の例}
    $\Ccal, \Dcal, \Ecal$: 圏

    % 関手圏から関手圏への関手として「関手$F$を前から水平合成する」「関手$F$を後ろから水平合成する」という操作を考える.
    \fbox{例2.25} 関手$F \colon \Ccal \to \Dcal$に対して,関手$- \bullet F \colon \Ecal^\Dcal \to \Ecal^\Ccal$を定義
    \begin{exampleblock}{\fbox{例2.32} 対角関手}
        今度は$\Ccal = \onebf$と取る.
        \begin{itemize}
            \item $F \colon \Ccal \to \onebf$は$\Ccal$の全ての対象を$\onebf$のただ1つの対象$* \in \onebf$へ写す
                \begin{itemize}
                    \item そもそも$\Ccal \to \onebf$の関手は1つしかないので$!$で表す.
                \end{itemize}
            \item $\Ecal^\onebf$は$\Ecal$と同型 (cf. 例2.23)
        \end{itemize}
        この関手$- \bullet {!} \colon \Ecal \to \Ecal^\Ccal$を\structure{対角関手} (diagnal functor) と呼び,$\Delta_\Ccal$でも表す.
    \end{exampleblock}
    すなわち,この$\Delta_\Ccal$は次のような関手.
        \begin{description}[対象への作用:]
            \item[対象への作用:] $\Ecal$の各対象$a$を関手$a \bullet {!} \in \Ecal^\Ccal$に写す
            \item[射への作用:] $\Ecal$の各射\begin{tikzcd}a \arrow[r, "f"] & b\end{tikzcd}を自然変換$f \bullet {!} = \qty{f}_{c \in \Ccal} \in \Ecal^\Ccal(a \bullet {!}, b \bullet {!})$に写す.
        \end{description}
\end{frame}
\end{document}
