\usepackage{pxjahyper}
\usepackage{amsmath,amsfonts,amssymb,amsthm,bm,ascmac}
\usepackage{ascmac}
\usepackage{physics}
\usepackage{tikz}
\usepackage{gnuplot-lua-tikz}
\usetikzlibrary{intersections,calc,arrows.meta,cd,automata,positioning}
\usepackage{circuitikz}
\usepackage{here}
\usepackage{siunitx}
\usepackage{multicol,multirow}
\usepackage{systeme}
\usepackage[version=3]{mhchem}
\usepackage{chemfig}
\usepackage{url}
\usepackage{braket}
\usepackage{enumerate}
\usepackage{mathrsfs}
\usepackage{otf}
\usepackage{ulem}
\usepackage{stmaryrd}
\usepackage{listings}
\usepackage{bussproofs}
\usepackage{mathtools}
\usepackage{cite}
\usepackage{docmute}
\DeclareMathOperator{\Sinarc}{\mathrm{Sin}^{-1}}
\DeclareMathOperator{\Cosarc}{\mathrm{Cos}^{-1}}
\DeclareMathOperator{\Tanarc}{\mathrm{Tan}^{-1}}
\DeclareMathOperator{\Real}{\mathbb{R}}
\DeclareMathOperator{\Complex}{\mathbb{C}}
\DeclareMathOperator{\Rational}{\mathbb{Q}}
\DeclareMathOperator{\Natural}{\mathbb{N}}
\DeclareMathOperator{\Integer}{\mathbb{Z}}
\DeclareMathOperator{\Ker}{\mathrm{Ker}}
\DeclareMathOperator{\diffset}{\backslash}
\DeclareMathOperator{\define}{\overset{\text{def}}{\Longleftrightarrow}}
\newcommand{\typed}{\!:\!}
\newcommand{\Var} {\mathord{\mathbf{Var}}}
\newcommand{\TVar}{\mathord{\mathbf{TVar}}}
\newcommand{\Type}{\mathord{\mathbf{Type}}}
\newcommand{\Term}{\mathord{\mathbf{\Lambda}}}
\newcommand{\Asn}{\mathord{\mathbf{Asn}}}
\newcommand{\Cont}{\mathord{\mathbf{Cont}}}
\newcommand{\Church}{\mathord{\mathbf{M}}}
\newcommand{\Sub}  {\mathord{\mathrm{Sub}}}
\newcommand{\FV}  {\mathord{\mathrm{FV}}}
\newcommand{\BV}  {\mathord{\mathrm{BV}}}
\newcommand{\vars}{\mathcal{V}}
\newcommand{\tvars}{\mathbb{V}}
\newcommand{\terms}{\mathcal{T}}
\newcommand{\types}{\mathbb{T}}
\newcommand{\contexts}{\mathfrak{G}}
\newcommand{\ID}  {(\mathrm{ID})}
\newcommand{\WEAK}{(\mathrm{WEAK})}
\newcommand{\EXC}{(\mathrm{EXC})}
\newcommand{\VAR} {(\mathrm{VAR})}
\newcommand{\APP} {(\mathrm{APP})}
\newcommand{\BND} {(\mathrm{BND})}
\newcommand{\INST}{(\mathrm{INST})}
\newcommand{\GEN} {(\mathrm{GEN})}
\newcommand{\tbeta}{\to_\beta}
\newcommand{\ttbeta}{\twoheadrightarrow_\beta}
\newcommand{\dom}{\mathord{\mathrm{dom}}}
\newcommand{\cod}{\mathord{\mathrm{cod}}}
\newcommand{\evf}{{\mathrm{ev}}}
\newcommand{\Typ}{\mathord{\mathrm{Typ}}}
\newcommand{\SN}{\mathsf{SN}}
\newcommand{\SAT}{\mathsf{SAT}}
\newcommand{\Bool}{\mathbb{B}}
\newcommand{\TCP}{\mathbf{TCP}}
\newcommand{\SUP}{\mathbf{SUP}}
\newcommand{\Ltt}{\mathtt{L}}
\newcommand{\Rtt}{\mathtt{R}}
\newcommand{\Stt}{\mathtt{S}}
% \newcommand{\init}{\mathrm{init}}
\newcommand{\init}{0}
\newcommand{\acc}{\mathrm{acc}}
\newcommand{\rej}{\mathrm{rej}}
\newcommand{\bak}{\mathrm{bak}}
\newcommand{\textspace}{\text{\textvisiblespace}\,}
\newcommand{\terminate}{\!\downarrow}
\newcommand{\nterminate}{\!\uparrow}
\newcommand{\problems}{\mathcal{P}_\Sigma}
\newcommand{\classified}{\in}
\newcommand{\Halt}{\mathtt{Halt}}
\newcommand{\Ccal}{\mathcal{C}}
\newcommand{\Dcal}{\mathcal{D}}
\newcommand{\Ecal}{\mathcal{E}}
\newcommand{\onebf}{{\mathord{\mathbf{1}}}}
\newcommand{\twobf}{{\mathord{\mathbf{2}}}}
\newcommand{\onebb}{{\mathord{\mathit{1}}}}
\newcommand{\twobb}{{\mathord{\mathit{2}}}}
\newcommand{\Cat}{\mathbf{Cat}}
\newcommand{\CAT}{\mathbf{CAT}}

\DeclarePairedDelimiter{\interpret}{\llbracket}{\rrbracket}
\DeclarePairedDelimiter{\encode}{\langle}{\rangle}

\DeclareMathOperator{\betared}{\rightarrow_\beta}
\DeclareMathOperator{\lbetared}{\longrightarrow_\beta}
\DeclareMathOperator{\tbetared}{\twoheadrightarrow_\beta}
\newcommand{\nat}{\mathtt{nat}}
\newcommand{\snat}{\nat_\sigma}
\newcommand{\CSet}{{\mathbf{Set}}}
\newcommand{\DSet}{{\mathbf{DSet}}}
\DeclareMathOperator{\Hom}{{\mathrm{Hom}}}
\DeclareMathOperator{\ob}{{\mathrm{ob}}}
\DeclareMathOperator{\mor}{{\mathrm{mor}}}
\newcommand{\lamto}{{\lambda\!\!\to}}
\newcommand{\lamP}{\lambda\mathrm{P}}
\newcommand{\lamT}{\lambda 2}
\newcommand{\lamPT}{\lambda\mathrm{P}2}
\newcommand{\lamo}{\lambda\underline{\omega}}
\newcommand{\lamPo}{\lambda\mathrm{P}\underline{\omega}}
\newcommand{\lamO}{\lambda\omega}
\newcommand{\lamC}{\lambda\mathrm{C}}

\DeclareMathOperator{\pterm}{\mathbf{Pt}}
\newcommand{\PROP}{\textbf{PROP}}
\newcommand{\PRED}{\textbf{PRED}}
\newcommand{\PROPT}{\textbf{PROP2}}
\newcommand{\PREDT}{\textbf{PRED2}}
\newcommand{\PROPo}{\textbf{PROP}\underline{\omega}}
\newcommand{\PREDo}{\textbf{PRED}\underline{\omega}}
\newcommand{\PROPO}{\textbf{PROP}\omega}
\newcommand{\PREDO}{\textbf{PRED}\omega}

\newcommand{\Yes}{\mathtt{Yes}}
\newcommand{\No}{\mathtt{No}}

\theoremstyle{definition}
\setbeamertemplate{theorems}[numbered]
\newtheorem{dfn}{定義}
\newtheorem*{ndfn}{定義}
\newtheorem{axiom}{公理}
\newtheorem{thm}{定理}
\newtheorem{prop}{命題}
\newtheorem{lem}{補題}
\newtheorem{cor}{系}
\newtheorem{ex}{例}
\newtheorem{question}{問}
\newtheorem{caution}{注意}
\newtheorem{notation}{記法}
\newtheorem{conjecture}{予想}


\usepackage{pifont}
\lstset{
    language={C++}, %プログラミング言語によって変える。
    basicstyle={\ttfamily\small},
    %% breaklines=true, %折り返し
}
\lstset{
    basicstyle={\ttfamily},
    identifierstyle={\small},
    commentstyle={\smallitshape},
    keywordstyle={\color{blue}\small\bfseries},
    commentstyle={\color{gray}\small},
    stringstyle={\color{red}\small},
    tabsize=2,
    frame={tb},
    breaklines=true,
    columns=[l]{fullflexible},
    numbers=left,
    xrightmargin=0zw,
    xleftmargin=3zw,
    numberstyle={\scriptsize},
    stepnumber=1,
    numbersep=1zw,
    lineskip=-0.5ex
}

\newenvironment<>{varblock}[2][\textwidth]{%
    \setlength{\textwidth}{#1}
    \begin{actionenv}#3%
        \def\insertblocktitle{#2}%
        \par%
    \usebeamertemplate{block begin}}
    {\par%
        \usebeamertemplate{block end}%
\end{actionenv}}

\newenvironment<>{varalertblock}[2][\textwidth]{%
    \setlength{\textwidth}{#1}
    \begin{actionenv}#3%
        \def\insertblocktitle{#2}%
        \par%
    \usebeamertemplate{block alerted begin}}
    {\par%
        \usebeamertemplate{block alerted end}%
\end{actionenv}}

\newenvironment{scprooftree}[1]%
  {\gdef\scalefactor{#1}\begin{center}\proofSkipAmount \leavevmode}%
  {\scalebox{\scalefactor}{\DisplayProof}\proofSkipAmount \end{center} }

\usetheme{Madrid}
% \usecolortheme{crane}
\usecolortheme{default}
\useinnertheme{}
\useoutertheme{}
\renewcommand{\kanjifamilydefault}{\gtdefault}
\usefonttheme{professionalfonts}
\setbeamertemplate{items}[default]
\setbeamertemplate{navigation symbols}{}
\renewcommand{\baselinestretch}{1}
\setbeamercolor{alerted text}{fg=orange!50!red}

\newcommand{\vcenterinput}[1]{\vcenter{\hbox{\input{#1}}}}
